\documentclass[journal,12pt,twocolumn]{IEEEtran}
%
\usepackage{setspace}
\usepackage{textcomp}
\usepackage{gensymb}
%\doublespacing
\singlespacing

\usepackage[cmex10]{amsmath}
\usepackage{amsthm}
%\usepackage{iithtlc}
\usepackage{mathrsfs}
\usepackage{txfonts}
\usepackage{stfloats}
\usepackage{bm}
\usepackage{cite}
\usepackage{cases}
\usepackage{subfig}
%\usepackage{xtab}
\usepackage{longtable}
\usepackage{multirow}
%\usepackage{algorithm}
%\usepackage{algpseudocode}
\usepackage{enumitem}
\usepackage{mathtools}
\usepackage{steinmetz}
\usepackage{tikz}
\usepackage{circuitikz}
\usepackage{verbatim}
\usepackage{tfrupee}
\usepackage[breaklinks=true]{hyperref}
%\usepackage{stmaryrd}
\usepackage{tkz-euclide} % loads  TikZ and tkz-base
%\usetkzobj{all}
\usetikzlibrary{calc,math}
\usepackage{listings}
    \usepackage{color}                                            %%
    \usepackage{array}                                            %%
    \usepackage{longtable}                                        %%
    \usepackage{calc}                                             %%
    \usepackage{multirow}                                         %%
    \usepackage{hhline}                                           %%
    \usepackage{ifthen}                                           %%
  %optionally (for landscape tables embedded in another document): %%
    \usepackage{lscape}     
\usepackage{multicol}
\usepackage{chngcntr}
%\usepackage{enumerate}

%\usepackage{wasysym}
%\newcounter{MYtempeqncnt}
\DeclareMathOperator*{\Res}{Res}
%\renewcommand{\baselinestretch}{2}
\renewcommand\thesection{\arabic{section}}
\renewcommand\thesubsection{\thesection.\arabic{subsection}}
\renewcommand\thesubsubsection{\thesubsection.\arabic{subsubsection}}

\renewcommand\thesectiondis{\arabic{section}}
\renewcommand\thesubsectiondis{\thesectiondis.\arabic{subsection}}
\renewcommand\thesubsubsectiondis{\thesubsectiondis.\arabic{subsubsection}}

% correct bad hyphenation here
\hyphenation{op-tical net-works semi-conduc-tor}
\def\inputGnumericTable{}                                 %%

\lstset{
%language=C,
frame=single, 
breaklines=true,
columns=fullflexible
}
\newenvironment{amatrix}[1]{%
  \left(\begin{array}{@{}*{#1}{c}|c@{}}
}{%
  \end{array}\right)
}
\DeclarePairedDelimiter\abs{\lvert}{\rvert}%
\DeclarePairedDelimiter\norm{\lVert}{\rVert}%

% Swap the definition of \abs* and \norm*, so that \abs
% and \norm resizes the size of the brackets, and the 
% starred version does not.
\makeatletter
\let\oldabs\abs
\def\abs{\@ifstar{\oldabs}{\oldabs*}}
%
\let\oldnorm\norm
\def\norm{\@ifstar{\oldnorm}{\oldnorm*}}
\makeatother

\newtheorem{theorem}{Theorem}[section]
\newtheorem{problem}{Problem}
\newtheorem{proposition}{Proposition}[section]
\newtheorem{lemma}{Lemma}[section]
\newtheorem{corollary}[theorem]{Corollary}
\newtheorem{example}{Example}[section]
\newtheorem{definition}[problem]{Definition}
%\newtheorem{thm}{Theorem}[section] 
%\newtheorem{defn}[thm]{Definition}
%\newtheorem{algorithm}{Algorithm}[section]
%\newtheorem{cor}{Corollary}
\newcommand{\BEQA}{\begin{eqnarray}}
\newcommand{\EEQA}{\end{eqnarray}}
\newcommand{\define}{\stackrel{\triangle}{=}}
\bibliographystyle{IEEEtran}
%\bibliographystyle{ieeetr}
\providecommand{\mbf}{\mathbf}
\providecommand{\pr}[1]{\ensuremath{\Pr\left(#1\right)}}
\providecommand{\qfunc}[1]{\ensuremath{Q\left(#1\right)}}
\providecommand{\sbrak}[1]{\ensuremath{{}\left[#1\right]}}
\providecommand{\lsbrak}[1]{\ensuremath{{}\left[#1\right.}}
\providecommand{\rsbrak}[1]{\ensuremath{{}\left.#1\right]}}
\providecommand{\brak}[1]{\ensuremath{\left(#1\right)}}
\providecommand{\lbrak}[1]{\ensuremath{\left(#1\right.}}
\providecommand{\rbrak}[1]{\ensuremath{\left.#1\right)}}
\providecommand{\cbrak}[1]{\ensuremath{\left\{#1\right\}}}
\providecommand{\lcbrak}[1]{\ensuremath{\left\{#1\right.}}
\providecommand{\rcbrak}[1]{\ensuremath{\left.#1\right\}}}
\providecommand{\system}{\overset{\mathcal{H}}{ \longleftrightarrow}}
	%\newcommand{\solution}[2]{\textbf{Solution:}{#1}}
\newcommand{\solution}{\noindent \textbf{Solution: }}
\newcommand{\cosec}{\,\text{cosec}\,}
\providecommand{\dec}[2]{\ensuremath{\overset{#1}{\underset{#2}{\gtrless}}}}
\newcommand{\myvec}[1]{\ensuremath{\begin{pmatrix}#1\end{pmatrix}}}
\newcommand{\mydet}[1]{\ensuremath{\begin{vmatrix}#1\end{vmatrix}}}
%\numberwithin{equation}{section}
\numberwithin{equation}{subsection}
%\numberwithin{problem}{section}
%\numberwithin{definition}{section}
\makeatletter
\@addtoreset{figure}{problem}
\makeatother
\let\StandardTheFigure\thefigure
\let\vec\mathbf
\usepackage{mathtools, nccmath}
\begin{document}
\begin{center}
\huge Assignment 7\\
\large SUBHASISH SAIKIA\\
\large AI20MTECH14001\\
\end{center}
\vspace{0.5cm}
\begin{abstract}
This document explains whether two lines meet at a point and to find out the distance of the point from any plane using singular valued decomposition and to verify the result using least squares.
\end{abstract}
\vspace{0.5cm}
Download all python codes from 
\begin{lstlisting}
https://github.com/subhasishsaikia22/EE5609-Matrix-theory
\end{lstlisting}
%
and latex-tikz codes from 
\begin{lstlisting}
https://github.com/subhasishsaikia22/EE5609-Matrix-theory
\end{lstlisting}
%
\vspace{0.5cm}
\section{Problem}
Show that the lines
\begin{align}
x&=4-2t,\quad y=-3+2t,\quad z=5-3t\\
x&=t,\quad y=1-4t,\quad z=-1+3t
\end{align}
meet in a point.\quad Determine the coordinates of this point.
Find the distance of the point from any one of the planes using single valued decomposition and verify the result using least squares.
\section{Point of Intersection of Two Lines}
Let the two lines
\begin{align}
L_1 : x=4-2t,\quad y=-3+2t,\quad z=5-3t\\
\implies L_1  :\vec{x}=\myvec{4\\-3\\5}+\lambda_1\myvec{-2\\2\\-3}\\
L_2  :  x=t,\quad y=1-4t,\quad z=-1+3t\\
\implies L_2  :\vec{x}=\myvec{0\\1\\-1}+\lambda_2\myvec{1\\-4\\3}
\end{align}
The two lines will meet at a point if and only if the rank of the coefficient matrix and the  rank of the augmented matrix are both 2.\\
If the two given lines intersect:
\begin{align}
\vec{x_1}+\lambda_1\vec{m_1}=\vec{x_2}+\lambda_2\vec{m_2}\\
\implies \myvec{m_1 \quad m_2}\myvec{\lambda_1\\-\lambda_2}=\vec{x_2}-\vec{x_1}\\
\implies \vec{M}\myvec{\lambda_1\\-\lambda_2}=\vec{x_2}-\vec{x_1}\label{eqn}
\intertext{where}
\vec{x_1}=\myvec{4\\-3\\5},\vec{x_2}=\myvec{0\\1\\-1},\vec{m_1}=\myvec{-2\\2\\-3},\vec{m_2}=\myvec{1\\-4\\3}\\
\vec{M}=\myvec{m_1 \quad m_2}
\end{align}
\eqref{eqn} can be expressed as the matrix equation:
\begin{align}
    \myvec{-2\quad1\\2\quad -4\\-3\quad 3}\vec{\lambda}=\myvec{-4\\4\\-6}
\end{align}
the augmented matrix of \eqref{eqn}
\begin{align}
    \myvec{-2 & 1  & -4 \\ 2 & -4 &  4\\ -3 & 3  & -6}\label{eqaug}\xleftrightarrow{R_2=R_2+R_1}\myvec{-2&1&-4\\0&-3&0\\-3&3&-6}\\
    \xleftrightarrow{R_3=R_3+\frac{3}{2} R_2}\myvec{-2&1&-4\\0&-3&0\\0&-3&0}\xleftrightarrow{R_3=R_3- R_2}\myvec{-2&1&-4\\0&-3&0\\0&0&0}\label{augmat}
\end{align}
Since both coefficient matrix and the augmented matrix has $rank=2$, the given equation of lines meet at a point.\\ 
From\eqref{augmat}
\begin{align}
\myvec{-2&1&-4\\0&-3&0\\0&0&0}\xleftrightarrow[R_1=R_1+R_2]{R_2=\frac{1}{3}R_2}\myvec{-2&0&-4\\0&-1&0\\0&0&0}
\\\xleftrightarrow{R_1=\frac{1}{-2}R_1}\myvec{1&0&2\\0&-1&0\\0&0&0}\label{rrefeqn}
\end{align}
From\eqref{eqn} and\eqref{rrefeqn},
\begin{align}
 \lambda_1=2,\lambda_2=0  
\end{align}
Therefore the point of intersection of the given lines:
\begin{align}
 L_1 : \Vec{x}=\myvec{4\\-3\\5}+2\myvec{-2\\2\\-3}\\
 \implies \vec{x}=\myvec{0\\1\\-1}\label{pointcoor}
\end{align}
\section{Coordinates of the foot of the perpendicular}
To find the distance of the point of intersection of the given lines \eqref{pointcoor} from any plane. We need to find the foot of the perpendicular from the particular point to the given plane.
Let us consider an arbitrary plane 
\begin{align}
    \myvec{1\quad2\quad-2}\Vec{x}=9\label{planeeqn}
    \intertext{and the point of intersection is}
    \Vec{b}=\myvec{0\\1\\-1}
\end{align}
The plane is in the form of 
\begin{align}
  \vec{n}^T\vec{x} =c 
  \intertext{from\eqref{planeeqn},where}
  \vec{n}=\myvec{1\\2\\-2}\\
  \vec{x} = \myvec{x\\y\\z}\\
  c=9
\end{align}
Let $\Vec{m_1}$ and $\vec{m_2}$ be the orthogonal vectors to the normal vector $\vec{n}$. Let $\vec{m}=\myvec{a\\b\\c}$,then
\begin{align}
    \vec{m}^T\vec{n}=0\\
    \implies \myvec{a\quad b\quad c}\myvec{1\\2\\-2}=0\\
    \implies a+2b-2c=0\\
\intertext{putting a=1 and b=0 we get,}
\vec{m_1}=\myvec{1\\0\\\frac{1}{2}}
\intertext{putting a=0 and b=1 we get,}
\vec{m_2}=\myvec{0\\1\\1}
\end{align}
Now we solve the equation:
\begin{align}
  \vec{M}\vec{x}=\vec{b}\label{footcooreqn}
\end{align}
In order to solve \eqref{footcooreqn},have to perform single valued decomposition on $\vec{M}$ as follows:
\begin{align}
\vec{M}=\vec{U}\vec{S}\vec{V}^T\label{eq100}
\end{align}
Where the columns of $\vec{V}$ are the eigen vectors of $\vec{M}^T\vec{M}$ ,the columns of $\vec{U}$ are the eigen vectors of $\vec{M}\vec{M}^T$ and $\vec{S}$ is diagonal matrix of singular value of eigenvalues of $\vec{M}^T\vec{M}$.
\begin{align}
\vec{M}^T\vec{M}=\myvec{\frac{5}{4}&\frac{1}{2}\\\frac{1}{2}&2}\label{eqMTM}\\
\vec{M}\vec{M}^T=\myvec{1&0&\frac{1}{2}\\0&1&1\\\frac{1}{2}&1&\frac{5}{4}}
\end{align}
From \eqref{footcooreqn} putting \eqref{eq100} we get,
\begin{align}\label{eqX}
\vec{U}\vec{S}\vec{V}^T\vec{x} & = \vec{b}\\
\implies\vec{x} &= \vec{V}\vec{S_+}\vec{U^T}\vec{b}
\end{align}
Where $\vec{S_+}$ is Moore-Penrose Pseudo-Inverse of $\vec{S}$.Now, calculating eigen value of $\vec{M}\vec{M}^T$,
\begin{align}
\mydet{\vec{M}\vec{M}^T - \lambda\vec{I}} = 0\\
\implies\myvec{1-\lambda\quad0\quad\frac{1}{2}\\0\quad1-\lambda\quad1\\\frac{1}{2}\quad1\quad\frac{5}{2}-\lambda} =0\\
\implies-4\lambda^3+13 \lambda^2-9\lambda =0
\end{align}
Hence eigen values of $\vec{M}\vec{M}^T$ are,
\begin{align}
\lambda_1 =\frac{9}{4}\\
\lambda_2 = 1\\
\lambda_3 =0
\end{align}
Hence the eigen vectors of $\vec{M}\vec{M}^T$ are,
\begin{align}
\vec{u}_1=\myvec{\frac{2}{5}\\\frac{4}{5}\\1},
\vec{u}_2=\myvec{-2\\1\\0},
\vec{u}_3=\myvec{\frac{-1}{2}\\-1\\1}
\end{align}
Normalizing the eigen vectors we get,
\begin{align}
\vec{u}_1=\myvec{\frac{2}{\sqrt{45}}\\\frac{4}{\sqrt{45}}\\\frac{5}{\sqrt{45}}},
\vec{u}_2=\myvec{-\frac{2}{\sqrt{5}}\\\frac{1}{\sqrt{5}}\\0},
\vec{u}_3=\myvec{-\frac{1}{3}\\-\frac{2}{3}\\\frac{2}{3}}
\end{align}
Hence we obtain $\vec{U}$ of \eqref{eq100} as follows,
\begin{align}\label{eqU}
\myvec{\frac{2}{\sqrt{45}}& -\frac{2}{\sqrt{5}}&-\frac{1}{3}\\
\frac{4}{\sqrt{45}}&\frac{1}{\sqrt{5}}&-\frac{2}{3}\\
\frac{5}{\sqrt{45}}&0&\frac{2}{3}}
\end{align}
After computing the singular values from eigen values $\lambda_1, \lambda_2, \lambda_3$ we get $\vec{S}$ of \eqref{eq100} as follows,
\begin{align}\label{eqS}
\vec{S}=\myvec{\frac{3}{2}&0\\0&1\\0&0}
\end{align}
Now, calculating eigen value of $\vec{M}^T\vec{M}$,
\begin{align}
\mydet{\vec{M}^T\vec{M} - \lambda\vec{I}} &= 0\\
\implies\myvec{\frac{5}{4}-\lambda&\frac{1}{2}\\\frac{1}{2}&2-\lambda} &=0\\
\implies\lambda^2-\frac{13}{4}\lambda+\frac{9}{4} &=0
\end{align}
Hence eigen values of $\vec{M}^T\vec{M}$ are,
\begin{align}
\lambda_4 &= \frac{9}{4}\\
\lambda_5 &=1
\end{align}
Hence the eigen vectors of $\vec{M}^T\vec{M}$ are,
\begin{align}
\vec{v}_1=\myvec{\frac{1}{2}\\1},
\vec{v}_2=\myvec{-2\\1}
\intertext{Normalizing the eigen vectors we get,}
\vec{v}_1=\myvec{\frac{1}{\sqrt{5}}\\\frac{2}{\sqrt{5}}},
\vec{v}_2=\myvec{-\frac{2}{\sqrt{5}}\\\frac{1}{\sqrt{5}}}
\end{align}
Hence we obtain $\vec{V}$ of \eqref{eq100} as follows,
\begin{align}
\vec{V}=\myvec{\frac{1}{\sqrt{5}}&-\frac{2}{\sqrt{5}}\\\frac{2}{\sqrt{5}}&\frac{1}{\sqrt{5}}}
\end{align}
 From \eqref{eq100} we get the Singular Value Decomposition of $\vec{M}$ ,
\begin{align}
\vec{M} = \myvec{\frac{2}{\sqrt{45}}&-\frac{2}{\sqrt{5}}&-\frac{1}{3}\\
\frac{4}{\sqrt{45}} & \frac{1}{\sqrt{5}}&\frac{-2}{3}\\
\frac{5}{\sqrt{45}}  & 0&\frac{2}{3}}\myvec{\frac{3}{2}&0\\0&1\\0&0}\myvec{\frac{1}{\sqrt{5}}&-\frac{2}{\sqrt{5}}\\\frac{2}{\sqrt{5}}&\frac{1}{\sqrt{5}}}^T
\end{align}
Moore-Penrose Pseudo inverse of $\vec{S}$ is given by,
\begin{align}
\vec{S_+} = \myvec{\frac{2}{3}&0&0\\0&1&0}
\end{align}
From \eqref{eqX} we get,
\begin{align}
\vec{U}^T\vec{b}&=\myvec{-\frac{1}{\sqrt{45}}\\\frac{1}{\sqrt{5}}\\\frac{-4}{3}}\\
\vec{S_+}\vec{U}^T\vec{b}=\myvec{-\frac{2}{3\sqrt{45}}\\\frac{\sqrt{1}}{\sqrt{5}}}\\
\vec{x} = \vec{V}\vec{S_+}\vec{U}^T\vec{b} = \myvec{-\frac{4}{9}\\\frac{1}{9}}\label{eq85}
\end{align}
\section{Verification using least square}
Verifying the solution of \eqref{eq85} using,
\begin{align}
\vec{M}^T\vec{M}\vec{x} = \vec{M}^T\vec{b}\label{eqVerify}
\end{align}
Evaluating the R.H.S in \eqref{eqVerify} we get,
\begin{align}
\vec{M}^T\vec{M}\vec{x} = \myvec{-\frac{1}{2}\\0}\\
\implies\myvec{\frac{5}{4}&\frac{1}{2}\\\frac{1}{2}&2}\vec{x} = \myvec{-\frac{1}{2}\\0}\label{eq:eq17}
\end{align}
Solving the augmented matrix of \eqref{eq:eq17} we get,
\begin{align}
\myvec{\frac{5}{4}&\frac{1}{2}&-\frac{1}{2}\\\frac{1}{2}&2&0} &\xleftrightarrow{R_2=R_2-\frac{2}{5}R_1}\myvec{\frac{5}{4}&\frac{1}{2}&\frac{-1}{2}\\0&\frac{9}{5}&\frac{1}{5}}\\
\xleftrightarrow[R_1=\frac{4}{5}R_1]  {R_2=\frac{5}{9}R_2}\myvec{1&\frac{2}{5}&\frac{-2}{5}\\0&1&\frac{1}{9}}&\xleftrightarrow{R_1=R_1-\frac{2}{5}R_2}\myvec{1&0&-\frac{4}{9}\\0&1&\frac{1}{9}}\label{eq:eq13}
\end{align}
From equation \eqref{eq:eq13}, solution is given by,
\begin{align}\label{eq:eq14}
\vec{x}=\myvec{\frac{-4}{9}\\\frac{1}{9}}
\end{align}
Comparing results of $\vec{x}$ from \eqref{eq85} and \eqref{eq:eq14}, we can say that the solution is verified\\

\section{Distance from a point to a plane}
Let the point of intersection of the lines  be represented by\quad$\vec{P}$ and the foot of the perpendicular from the particular point\quad $\vec{P}$ to the given plane be $\vec{Q}$.`\\
From  \eqref{pointcoor}and \eqref{eq85} ,
\begin{align}
    \vec{P}=\myvec{0\\1\\-1}\\
    \vec{Q}=\myvec{\frac{-4}{9}\\\frac{1}{9}\\0}
\end{align}
The distance between the point of intersection  of the lines from  the plane is:
\begin{align}
    \norm{\vec{P}-\vec{Q}}=\sqrt{\brak{0-\frac{-4}{9}}^2+\brak{1-\frac{1}{9}}^2+\brak{-1-0}^2}\\
\implies \norm{\vec{P}-\vec{Q}}=\frac{\sqrt{161}}{9}=1.409
\end{align}
\end{document}
