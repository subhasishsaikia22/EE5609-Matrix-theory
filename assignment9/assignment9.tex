\documentclass[journal,12pt]{IEEEtran}
\usepackage{longtable}
\usepackage{setspace}
\usepackage{gensymb}
\singlespacing
\usepackage[cmex10]{amsmath}
\newcommand\myemptypage{
\null
\thispagestyle{empty}
\addtocounter{page}{-1}
\newpage
}
\usepackage{amsthm}
\usepackage{mdframed}
\usepackage{mathrsfs}
\usepackage{txfonts}
\usepackage{stfloats}
\usepackage{bm}
\usepackage{cite}
\usepackage{cases}
\usepackage{subfig}

\usepackage{longtable}
\usepackage{multirow}

\usepackage{enumitem}
\usepackage{mathtools}
\usepackage{steinmetz}
\usepackage{tikz}
\usepackage{circuitikz}
\usepackage{verbatim}
\usepackage{tfrupee}
\usepackage[breaklinks=true]{hyperref}
\usepackage{graphicx}
\usepackage{tkz-euclide}

\usetikzlibrary{calc,math}
\usepackage{listings}
    \usepackage{color}                                            %%
    \usepackage{array}                                            %%
    \usepackage{longtable}                                        %%
    \usepackage{calc}                                             %%
    \usepackage{multirow}                                         %%
    \usepackage{hhline}                                           %%
    \usepackage{ifthen}                                           %%
    \usepackage{lscape}    
\usepackage{multicol}
\usepackage{chngcntr}

\DeclareMathOperator*{\Res}{Res}

\renewcommand\thesection{\arabic{section}}
\renewcommand\thesubsection{\thesection.\arabic{subsection}}
\renewcommand\thesubsubsection{\thesubsection.\arabic{subsubsection}}

\renewcommand\thesectiondis{\arabic{section}}
\renewcommand\thesubsectiondis{\thesectiondis.\arabic{subsection}}
\renewcommand\thesubsubsectiondis{\thesubsectiondis.\arabic{subsubsection}}


\hyphenation{op-tical net-works semi-conduc-tor}
\def\inputGnumericTable{}                                 %%

\lstset{
%language=C,
frame=single,
breaklines=true,
columns=fullflexible
}
\begin{document}
\onecolumn

\newtheorem{theorem}{Theorem}[section]
\newtheorem{problem}{Problem}
\newtheorem{proposition}{Proposition}[section]
\newtheorem{lemma}{Lemma}[section]
\newtheorem{corollary}[theorem]{Corollary}
\newtheorem{example}{Example}[section]
\newtheorem{definition}[problem]{Definition}

\newcommand{\BEQA}{\begin{eqnarray}}
\newcommand{\EEQA}{\end{eqnarray}}
\newcommand{\define}{\stackrel{\triangle}{=}}
\bibliographystyle{IEEEtran}
\raggedbottom
\setlength{\parindent}{0pt}
\providecommand{\mbf}{\mathbf}
\providecommand{\pr}[1]{\ensuremath{\Pr\left(#1\right)}}
\providecommand{\qfunc}[1]{\ensuremath{Q\left(#1\right)}}
\providecommand{\sbrak}[1]{\ensuremath{{}\left[#1\right]}}
\providecommand{\lsbrak}[1]{\ensuremath{{}\left[#1\right.}}
\providecommand{\rsbrak}[1]{\ensuremath{{}\left.#1\right]}}
\providecommand{\brak}[1]{\ensuremath{\left(#1\right)}}
\providecommand{\lbrak}[1]{\ensuremath{\left(#1\right.}}
\providecommand{\rbrak}[1]{\ensuremath{\left.#1\right)}}
\providecommand{\cbrak}[1]{\ensuremath{\left\{#1\right\}}}
\providecommand{\lcbrak}[1]{\ensuremath{\left\{#1\right.}}
\providecommand{\rcbrak}[1]{\ensuremath{\left.#1\right\}}}
\theoremstyle{remark}
\newtheorem{rem}{Remark}
\newcommand{\sgn}{\mathop{\mathrm{sgn}}}
%\providecommand{\hilbert}{\overset{\mathcal{H}}{ \rightleftharpoons}}
\providecommand{\system}{\overset{\mathcal{H}}{ \longleftrightarrow}}
%\newcommand{\solution}[2]{\textbf{Solution:}{#1}}
\newcommand{\solution}{\noindent \textbf{Solution: }}
\newcommand{\cosec}{\,\text{cosec}\,}
\providecommand{\dec}[2]{\ensuremath{\overset{#1}{\underset{#2}{\gtrless}}}}
\newcommand{\myvec}[1]{\ensuremath{\begin{pmatrix}#1\end{pmatrix}}}
\newcommand{\mydet}[1]{\ensuremath{\begin{vmatrix}#1\end{vmatrix}}}
\numberwithin{equation}{subsection}
\makeatletter
\@addtoreset{figure}{problem}
\makeatother
\let\StandardTheFigure\thefigure
\let\vec\mathbf
\renewcommand{\thefigure}{\theproblem}
\def\putbox#1#2#3{\makebox[0in][l]{\makebox[#1][l]{}\raisebox{\baselineskip}[0in][0in]{\raisebox{#2}[0in][0in]{#3}}}}
     \def\rightbox#1{\makebox[0in][r]{#1}}
     \def\centbox#1{\makebox[0in]{#1}}
     \def\topbox#1{\raisebox{-\baselineskip}[0in][0in]{#1}}
     \def\midbox#1{\raisebox{-0.5\baselineskip}[0in][0in]{#1}}
\vspace{3cm}
\title{Assignment 9}
\author{Subhasish Saikia\\AI20MTECH14001}
\maketitle
\begin{abstract}
This document explains the properties of inner product on $\mathbb{R}^{n}$
\end{abstract}
\bigskip
\renewcommand{\thefigure}{\theenumi}
\renewcommand{\thetable}{\theenumi}
Download latex-tikz codes from
\begin{lstlisting}
https://github.com/subhasishsaikia22/EE5609-Matrix-theory
\end{lstlisting}
\section{\textbf{Problem}}
Let $\langle,\rangle :\mathbb{R}^n\times \mathbb{R}^n \rightarrow \mathbb{R}$  denote the standard inner product on $\mathbb{R}^{n}$. For a non zero $\vec{w} \in \mathbb{R}^n,$ define $T_\vec{w}:\mathbb{R}^n \rightarrow \mathbb{R}^n$ by $T_\vec{w}(\vec{v})=\vec{v}-\frac{2\langle \vec{v},\vec{w} \rangle}{\langle \vec{w},\vec{w} \rangle}\vec{w},\vec{v}\in \mathbb{R}^n$. Which of the following are true?\\
\begin{enumerate}
\item  det $\brak {T_\vec{w}}$=1 \\
\item $\langle T_\vec{w}\brak{\vec{v_1}},T_\vec{w}\brak{\vec{v_2}}\rangle=\langle \vec{v_1},\vec{v_2}\rangle \quad \forall \vec{v_1}$ \\ 
\item  $T_\vec{w}={T_\vec{w}^{-1}}$\\
\item  $T_\vec{2w}=2T_\vec{w}$\\
\end{enumerate}
\section{\textbf{Explanation}}
\renewcommand{\thetable}{1}
\begin{longtable}{|l|l|}
\hline
\textbf{Inner Product} & \text{Let two vectors $\vec{u}$ and $\vec{v}$ be defined as:}\\& \parbox{13cm} {\begin{align}
    \vec{u}=\myvec{{u_1}\\\vec^{u_2}\\.\\.\\.\\.\\\vec{u_n}},\vec{v}=\myvec{{v_1}\\\vec^{v_2}\\.\\.\\.\\.\\\vec{v_n}}\in \mathbb{R}^n
\end{align}}\\
& \text{then the  inner product of $\vec{u}$ and $\vec{v}$ on $\mathbb{R}^n$:}\\ & \parbox{10cm} {\begin{align}
   \langle \vec{u},\vec{v} \rangle=\vec{u}^T \vec{v}= \vec{u_1}\vec{v_1}+\vec{u_2}\vec{v_2}+...+\vec{u_n}\vec{v_n}
   \end{align}}\\
 \hline
\textbf{Inner Product}&\\\textbf{Property used}&\parbox{13cm}{\begin{align}
    \langle\vec{x},\vec{y}\rangle=\vec{x}^T\vec{y}=\vec{y}^T\vec{x}=\langle\vec{y},\vec{x}\rangle\label{prop1}
    \end{align}}\\
     & \text{linearity property:}\\ & \parbox{10cm} {\begin{align}
  \langle a\vec{u}+b\vec{v},\vec{w}\rangle=a\langle \vec{u},\vec{w}\rangle+b\langle \vec{v},\vec{w}\rangle
\end{align}}\\
\hline
\caption{Definition and properties used}
\label{deftab}
\end{longtable}\\
\section{\textbf{Solution}}
\renewcommand{\thetable}{2}
\begin{longtable}{|l|l|}
\hline
\textbf{Given} & \text{For n=2,}\\& \parbox{13cm} {\begin{align}
 T_\vec{w}:\mathbb{R}^2 \rightarrow \mathbb{R}^2,T_\vec{w}\brak{\vec{v}}=\vec{v}-\frac{2\langle \vec{v},\vec{w} \rangle}{\langle \vec{w},\vec{w} \rangle}\vec{w}
\end{align}}\\
&\text{Let the standard basis vectors of $\mathbb{R}^2:\vec{e_1}=\myvec{1\\0},\vec{e_2}=\myvec{0\\1}$ and non zero vector $\vec{w}=\myvec{1\\1}$}\\
\hline
\textbf{Statement 1} & \text{det $\brak {T_\vec{w}}$=1}\\
\hline
\textbf{solution}   & \parbox{13cm} {\begin{align}
T_\vec{w}\brak{\vec{e_1}}=\vec{e_1}-\frac{2\langle \vec{e_1},\vec{w} \rangle}{\langle \vec{w},\vec{w} \rangle}\vec{w}=\myvec{1\\0}-\frac{2\langle \myvec{1\\0},\myvec{1\\1} \rangle}{\langle \myvec{1\\1},\myvec{1\\1} \rangle}\myvec{1\\1}=\myvec{1\\0}-\frac{2.1}{2}\myvec{1\\1}=\myvec{1\\0}-\myvec{1\\1}=\myvec{0\\-1}
\end{align}}\\
& \parbox{10cm} {\begin{align}
T_\vec{w}\brak{\vec{e_2}}=\vec{e_2}-\frac{2\langle \vec{e_2},\vec{w} \rangle}{\langle \vec{w},\vec{w} \rangle}\vec{w}=\myvec{0\\1}-\frac{2\langle \myvec{0\\1},\myvec{1\\1} \rangle}{\langle \myvec{1\\1},\myvec{1\\1} \rangle}\myvec{1\\1}=\myvec{0\\1}-\frac{2.1}{2}\myvec{1\\1}=\myvec{0\\1}-\myvec{1\\1}=\myvec{-1\\0}
\end{align}}\\
& \parbox{10cm} {\begin{align}
T_\vec{w}=\myvec{T_\vec{w}\brak{\vec{e_1}} &T_\vec{w}\brak{\vec{e_2}}}=\myvec{0 & -1\\-1&0}\\
\implies \mydet{{T_\vec{w}}}=\mydet{0 & -1\\-1&0}=0-1=-1
\end{align}}\\
& \parbox{10cm}{\begin{center}
\textbf{The given statement is false}
\end{center}}\\
\hline 
\textbf{Statement 2} & \text{$\bigg\langle T_\vec{w}\brak{\vec{v_1}},T_\vec{w}\brak{\vec{v_2}}\bigg\rangle=\langle \vec{v_1},\vec{v_2}\rangle \quad \forall \vec{v_1}$}\\
\hline
\textbf{solution}   & \parbox{13cm} {\begin{align}
\bigg\langle T_\vec{w}\brak{\vec{v_1}},T_\vec{w}\brak{\vec{v_2}}\bigg\rangle=\bigg\langle \vec{v_1}-\frac{2\langle \vec{v_1},\vec{w} \rangle}{\langle \vec{w},\vec{w} \rangle}\vec{w}, \vec{v_2}-\frac{2\langle \vec{v_2},\vec{w} \rangle}{\langle \vec{w},\vec{w} \rangle}\vec{w}\bigg\rangle\\
=\bigg\langle \vec{v_1},\vec{v_2}-\frac{2\langle \vec{v_2},\vec{w} \rangle}{\langle \vec{w},\vec{w} \rangle}\vec{w}\bigg\rangle+\bigg\langle-\frac{2\langle \vec{v_1},\vec{w} \rangle}{\langle \vec{w},\vec{w} \rangle}\vec{w},\vec{v_2}-\frac{2\langle \vec{v_2},\vec{w} \rangle}{\langle \vec{w},\vec{w} \rangle}\vec{w}\bigg\rangle\\
=\langle \vec{v_1},\vec{v_2}\rangle-\frac{2\langle \vec{v_2},\vec{w} \rangle}{\langle \vec{w}, \vec{w}\rangle}\langle\vec{v_1},\vec{w}\rangle+\bigg\langle-\frac{2\langle \vec{v_1},\vec{w} \rangle}{\langle \vec{w},\vec{w} \rangle}\vec{w},\vec{v_2}\bigg\rangle+\bigg\langle-\frac{2\langle \vec{v_1},\vec{w} \rangle}{\langle \vec{w},\vec{w} \rangle}\vec{w},-\frac{2\langle \vec{v_2},\vec{w} \rangle}{\langle \vec{w},\vec{w} \rangle}\vec{w}\bigg\rangle\\
=\langle \vec{v_1},\vec{v_2}\rangle-\frac{2\langle \vec{v_2},\vec{w} \rangle}{\langle \vec{w}, \vec{w}\rangle}\langle\vec{v_1},\vec{w}\rangle-\frac{2\langle \vec{v_1},\vec{w} \rangle}{\langle \vec{w},\vec{w} \rangle}\langle\vec{w},\vec{v_2}\rangle+\frac{4\langle\vec{v_1},\vec{w}\rangle \langle\vec{v_2},\vec{w}\rangle}{\langle \vec{w},\vec{w}\rangle\langle \vec{w},\vec{w}\rangle} {\langle\vec{w},\vec{w}\rangle}\\
=\langle \vec{v_1},\vec{v_2}\rangle-\frac{2\langle \vec{v_2},\vec{w} \rangle}{\langle \vec{w}, \vec{w}\rangle}\langle\vec{v_1},\vec{w}\rangle-\frac{2\langle \vec{v_1},\vec{w} \rangle}{\langle \vec{w},\vec{w} \rangle}\langle\vec{v_2},\vec{w}\rangle+\frac{4\langle\vec{v_1},\vec{w}\rangle \langle\vec{v_2},\vec{w}\rangle}{\langle \vec{w},\vec{w}\rangle}\\
\implies\bigg\langle T_\vec{w}\brak{\vec{v_1}},T_\vec{w}\brak{\vec{v_2}}\bigg\rangle=\langle \vec{v_1},\vec{v_2}\rangle
\end{align}}\\
& \parbox{13cm}{\begin{center}
\textbf{The given statement is correct}
\end{center}}\\
\hline 
\textbf{Statement 3} & \text{ $T_\vec{w}={T_\vec{w}^{-1}}$}\\
\hline
\textbf{solution} & \text{Suppose:}\\  & \parbox{10cm} {\begin{align}
T_\vec{w}\brak{\vec{v}}=\vec{u}\implies T_\vec{w}^{-1}\brak{\vec{u}}=\vec{v}
\end{align}}\\
& \text{Then :}\\& \parbox{10cm} {\begin{align}
T_\vec{w}\brak{\vec{v}}=\vec{u}\\
   \implies \vec{v}-\frac{2\langle \vec{v},\vec{w} \rangle}{\langle \vec{w},\vec{w} \rangle}\vec{w}=\vec{u}\label{eq14}\\
   \implies \bigg\langle\vec{v}-\frac{2\langle \vec{v},\vec{w} \rangle}{\langle \vec{w},\vec{w} \rangle}\vec{w}, \vec{w}\bigg\rangle=\langle \vec{u},\vec{w}\rangle\\
   \implies\langle \vec{v},\vec{w} \rangle-\frac{2\langle \vec{v},\vec{w} \rangle}{\langle \vec{w},\vec{w} \rangle}\langle\vec{w}, \vec{w}\rangle=\langle \vec{u},\vec{w}\rangle\\
   \implies -\langle \vec{v},\vec{w}\rangle=\langle \vec{u},\vec{w}\rangle \label{eq17}
\end{align}}\\
& \text{using \eqref{eq14}\quad and \eqref{eq17}}\\
& \parbox{10cm} {\begin{align}
T_\vec{w}^{-1}\brak{\vec{u}}=\vec{v}=\vec{u}+\frac{2\langle \vec{v},\vec{w} \rangle}{\langle \vec{w},\vec{w} \rangle}\vec{w} \\
\implies T_\vec{w}^{-1}\brak{\vec{u}}=\vec{u}-\frac{2\langle \vec{u},\vec{w} \rangle}{\langle \vec{w},\vec{w} \rangle}\vec{w}\\
\implies T_\vec{w}^{-1}\brak{\vec{u}}= T_\vec{w}\brak{\vec{u}}\\
\implies T_\vec{w}^{-1}= T_\vec{w}
\end{align}}\\
& \parbox{10cm}{\begin{center}
\textbf{The given statement is correct}
\end{center}}\\
\hline 
\textbf{Statement 4} & \text{$T_\vec{2w}=2T_\vec{w}$}\\ 
\hline
\textbf{solution}  & \parbox{10cm} {\begin{align}
T_\vec{2w}\brak{\vec{v}}=\vec{v}-\frac{2\langle \vec{v},\vec{2w} \rangle}{\langle \vec{2w},\vec{2w} \rangle}\vec{2w}\\
=\vec{v}-\frac{2.2.2\langle \vec{v},\vec{w} \rangle}{2.2\langle \vec{w},\vec{w} \rangle}\vec{w}\\
=\vec{v}-\frac{2\langle \vec{v},\vec{w} \rangle}{\langle \vec{w},\vec{w} \rangle}\vec{w}\\
2T_\vec{w}\brak{\vec{v}}=2\brak{\vec{v}-\frac{2\langle \vec{v},\vec{w} \rangle}{\langle \vec{w},\vec{w} \rangle}\vec{w}}\\
\implies2T_\vec{w}\brak{\vec{v}}\neq T_\vec{2w}\brak{\vec{v}}
\end{align}}\\
& \parbox{10cm}{\begin{center}
\textbf{The given statement is false}
\end{center}}\\
\hline
\caption{solution}
\label{deftab}
\end{longtable}
\end{document}
