\documentclass[journal,12pt]{IEEEtran}
\usepackage{longtable}
\usepackage{setspace}
\usepackage{gensymb}
\singlespacing
\usepackage[cmex10]{amsmath}
\newcommand\myemptypage{
\null
\thispagestyle{empty}
\addtocounter{page}{-1}
\newpage
}
\usepackage{amsthm}
\usepackage{mdframed}
\usepackage{mathrsfs}
\usepackage{txfonts}
\usepackage{stfloats}
\usepackage{bm}
\usepackage{cite}
\usepackage{cases}
\usepackage{subfig}

\usepackage{longtable}
\usepackage{multirow}

\usepackage{enumitem}
\usepackage{mathtools}
\usepackage{steinmetz}
\usepackage{tikz}
\usepackage{circuitikz}
\usepackage{verbatim}
\usepackage{tfrupee}
\usepackage[breaklinks=true]{hyperref}
\usepackage{graphicx}
\usepackage{tkz-euclide}

\usetikzlibrary{calc,math}
\usepackage{listings}
    \usepackage{color}                                            %%
    \usepackage{array}                                            %%
    \usepackage{longtable}                                        %%
    \usepackage{calc}                                             %%
    \usepackage{multirow}                                         %%
    \usepackage{hhline}                                           %%
    \usepackage{ifthen}                                           %%
    \usepackage{lscape}    
\usepackage{multicol}
\usepackage{chngcntr}

\DeclareMathOperator*{\Res}{Res}

\renewcommand\thesection{\arabic{section}}
\renewcommand\thesubsection{\thesection.\arabic{subsection}}
\renewcommand\thesubsubsection{\thesubsection.\arabic{subsubsection}}

\renewcommand\thesectiondis{\arabic{section}}
\renewcommand\thesubsectiondis{\thesectiondis.\arabic{subsection}}
\renewcommand\thesubsubsectiondis{\thesubsectiondis.\arabic{subsubsection}}


\hyphenation{op-tical net-works semi-conduc-tor}
\def\inputGnumericTable{}                                 %%

\lstset{
%language=C,
frame=single,
breaklines=true,
columns=fullflexible
}
\begin{document}
\onecolumn

\newtheorem{theorem}{Theorem}[section]
\newtheorem{problem}{Problem}
\newtheorem{proposition}{Proposition}[section]
\newtheorem{lemma}{Lemma}[section]
\newtheorem{corollary}[theorem]{Corollary}
\newtheorem{example}{Example}[section]
\newtheorem{definition}[problem]{Definition}

\newcommand{\BEQA}{\begin{eqnarray}}
\newcommand{\EEQA}{\end{eqnarray}}
\newcommand{\define}{\stackrel{\triangle}{=}}
\bibliographystyle{IEEEtran}
\raggedbottom
\setlength{\parindent}{0pt}
\providecommand{\mbf}{\mathbf}
\providecommand{\pr}[1]{\ensuremath{\Pr\left(#1\right)}}
\providecommand{\qfunc}[1]{\ensuremath{Q\left(#1\right)}}
\providecommand{\sbrak}[1]{\ensuremath{{}\left[#1\right]}}
\providecommand{\lsbrak}[1]{\ensuremath{{}\left[#1\right.}}
\providecommand{\rsbrak}[1]{\ensuremath{{}\left.#1\right]}}
\providecommand{\brak}[1]{\ensuremath{\left(#1\right)}}
\providecommand{\lbrak}[1]{\ensuremath{\left(#1\right.}}
\providecommand{\rbrak}[1]{\ensuremath{\left.#1\right)}}
\providecommand{\cbrak}[1]{\ensuremath{\left\{#1\right\}}}
\providecommand{\lcbrak}[1]{\ensuremath{\left\{#1\right.}}
\providecommand{\rcbrak}[1]{\ensuremath{\left.#1\right\}}}
\theoremstyle{remark}
\newtheorem{rem}{Remark}
\newcommand{\sgn}{\mathop{\mathrm{sgn}}}
%\providecommand{\hilbert}{\overset{\mathcal{H}}{ \rightleftharpoons}}
\providecommand{\system}{\overset{\mathcal{H}}{ \longleftrightarrow}}
%\newcommand{\solution}[2]{\textbf{Solution:}{#1}}
\newcommand{\solution}{\noindent \textbf{Solution: }}
\newcommand{\cosec}{\,\text{cosec}\,}
\providecommand{\dec}[2]{\ensuremath{\overset{#1}{\underset{#2}{\gtrless}}}}
\newcommand{\myvec}[1]{\ensuremath{\begin{pmatrix}#1\end{pmatrix}}}
\newcommand{\mydet}[1]{\ensuremath{\begin{vmatrix}#1\end{vmatrix}}}
\numberwithin{equation}{subsection}
\makeatletter
\@addtoreset{figure}{problem}
\makeatother
\let\StandardTheFigure\thefigure
\let\vec\mathbf
\renewcommand{\thefigure}{\theproblem}
\def\putbox#1#2#3{\makebox[0in][l]{\makebox[#1][l]{}\raisebox{\baselineskip}[0in][0in]{\raisebox{#2}[0in][0in]{#3}}}}
     \def\rightbox#1{\makebox[0in][r]{#1}}
     \def\centbox#1{\makebox[0in]{#1}}
     \def\topbox#1{\raisebox{-\baselineskip}[0in][0in]{#1}}
     \def\midbox#1{\raisebox{-0.5\baselineskip}[0in][0in]{#1}}
\vspace{3cm}
\title{Assignment 10}
\author{Subhasish Saikia\\AI20MTECH14001}
\maketitle
\begin{abstract}
This document explains the procedure of determining the rank of the given matrix.
\end{abstract}
\bigskip
\renewcommand{\thefigure}{\theenumi}
\renewcommand{\thetable}{\theenumi}
Download latex-tikz codes from
\begin{lstlisting}
https://github.com/subhasishsaikia22/EE5609-Matrix-theory
\end{lstlisting}
\section{\textbf{Problem}}
What is the rank of the following matrix?\\
$\myvec{1&1&1&1&1\\1&2&2&2&2\\1&2&3&3&3\\1&2&3&4&4\\1&2&3&4&5}$\\
\begin{enumerate}
\item  2 \\
\item 3 \\ 
\item  4\\
\item  5\\
\end{enumerate}
\section{\textbf{Explanation}}
\renewcommand{\thetable}{1}
\begin{longtable}{|l|l|}
\hline
\textbf{Rank of a matrix} & \text{The rank of the given matrix is determine by reducing it to row reduced echelon form.}\\
& \text{A matrix is in row echelon form if:}\\
&\text{ $>$all rows consisting of only zeroes are at the bottom.}\\
&\text{$>$the leading coefficient of a nonzero row is always strictly }\\
&\text{to the right of the leading coefficient of the row above it }\\ 
 \hline
\caption{Definition}
\label{deftab}
\end{longtable}\\
\section{\textbf{Solution}}
\renewcommand{\thetable}{2}
\begin{longtable}{|l|l|}
\hline
\textbf{Given} & \parbox{13cm} {\begin{align}
 \myvec{1&1&1&1&1\\1&2&2&2&2\\1&2&3&3&3\\1&2&3&4&4\\1&2&3&4&5}\xleftrightarrow[R_2=R_2-R_1]{R_3=R_3-R_1}\myvec{1&1&1&1&1\\0&1&1&1&1\\0&1&2&2&2\\1&2&3&4&4\\1&2&3&4&5}\\
 \xleftrightarrow[R_4=R_4-R_1]{R_5=R_5-R_1}\myvec{1&1&1&1&1\\0&1&1&1&1\\0&1&2&2&2\\0&1&2&3&3\\0&1&2&3&4}\xleftrightarrow[R_5=R_5-R_4]{R_4=R_4-R_3}\myvec{1&1&1&1&1\\0&1&1&1&1\\0&1&2&2&2\\0&0&0&1&1\\0&0&0&0&1}\\\xleftrightarrow[R_4=R_4-R_5]{R_3=R_3-R_3}\myvec{1&1&1&1&1\\0&1&1&1&1\\0&0&1&1&1\\0&0&0&1&0\\0&0&0&0&1}\xleftrightarrow[R_2=R_2-R_3]{R_3=R_3-R_4-R_5}\myvec{1&1&1&1&1\\0&1&0&0&0\\0&0&1&0&0\\0&0&0&1&0\\0&0&0&0&1}\\\xleftrightarrow{R_1=R_1-R_2-R_3-R_4-R_5}\myvec{1&0&0&0&0\\0&1&0&0&0\\0&0&1&0&0\\0&0&0&1&0\\0&0&0&0&1}
\end{align}}\\
&\text{Thus the rank of the given matrix is $5$}\\
\hline
\caption{RREF and Rank}
\label{deftab}
\end{longtable}\\
\section{\textbf{Solution}}
\renewcommand{\thetable}{3}
\begin{longtable}{|l|l|l|}
\hline
\textbf{Option}&\textbf{Solution}&\textbf{True/}\\&&\textbf{False}\\
\hline
1 &\text{$2$}&False
\hline
2 &\text{$3$}&False
\hline
3 &\text{$4$}&False
\hline
4 &\text{$5$}&True
\hline
\caption{correct option}
\label{deftab}
\end{longtable}
\end{document}
