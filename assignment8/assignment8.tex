\documentclass[journal,12pt]{IEEEtran}
\usepackage{longtable}
\usepackage{setspace}
\usepackage{gensymb}
\singlespacing
\usepackage[cmex10]{amsmath}
\newcommand\myemptypage{
\null
\thispagestyle{empty}
\addtocounter{page}{-1}
\newpage
}
\usepackage{amsthm}
\usepackage{mdframed}
\usepackage{mathrsfs}
\usepackage{txfonts}
\usepackage{stfloats}
\usepackage{bm}
\usepackage{cite}
\usepackage{cases}
\usepackage{subfig}

\usepackage{longtable}
\usepackage{multirow}

\usepackage{enumitem}
\usepackage{mathtools}
\usepackage{steinmetz}
\usepackage{tikz}
\usepackage{circuitikz}
\usepackage{verbatim}
\usepackage{tfrupee}
\usepackage[breaklinks=true]{hyperref}
\usepackage{graphicx}
\usepackage{tkz-euclide}

\usetikzlibrary{calc,math}
\usepackage{listings}
    \usepackage{color}                                            %%
    \usepackage{array}                                            %%
    \usepackage{longtable}                                        %%
    \usepackage{calc}                                             %%
    \usepackage{multirow}                                         %%
    \usepackage{hhline}                                           %%
    \usepackage{ifthen}                                           %%
    \usepackage{lscape}    
\usepackage{multicol}
\usepackage{chngcntr}

\DeclareMathOperator*{\Res}{Res}

\renewcommand\thesection{\arabic{section}}
\renewcommand\thesubsection{\thesection.\arabic{subsection}}
\renewcommand\thesubsubsection{\thesubsection.\arabic{subsubsection}}

\renewcommand\thesectiondis{\arabic{section}}
\renewcommand\thesubsectiondis{\thesectiondis.\arabic{subsection}}
\renewcommand\thesubsubsectiondis{\thesubsectiondis.\arabic{subsubsection}}


\hyphenation{op-tical net-works semi-conduc-tor}
\def\inputGnumericTable{}                                 %%

\lstset{
%language=C,
frame=single,
breaklines=true,
columns=fullflexible
}
\begin{document}
\onecolumn

\newtheorem{theorem}{Theorem}[section]
\newtheorem{problem}{Problem}
\newtheorem{proposition}{Proposition}[section]
\newtheorem{lemma}{Lemma}[section]
\newtheorem{corollary}[theorem]{Corollary}
\newtheorem{example}{Example}[section]
\newtheorem{definition}[problem]{Definition}

\newcommand{\BEQA}{\begin{eqnarray}}
\newcommand{\EEQA}{\end{eqnarray}}
\newcommand{\define}{\stackrel{\triangle}{=}}
\bibliographystyle{IEEEtran}
\raggedbottom
\setlength{\parindent}{0pt}
\providecommand{\mbf}{\mathbf}
\providecommand{\pr}[1]{\ensuremath{\Pr\left(#1\right)}}
\providecommand{\qfunc}[1]{\ensuremath{Q\left(#1\right)}}
\providecommand{\sbrak}[1]{\ensuremath{{}\left[#1\right]}}
\providecommand{\lsbrak}[1]{\ensuremath{{}\left[#1\right.}}
\providecommand{\rsbrak}[1]{\ensuremath{{}\left.#1\right]}}
\providecommand{\brak}[1]{\ensuremath{\left(#1\right)}}
\providecommand{\lbrak}[1]{\ensuremath{\left(#1\right.}}
\providecommand{\rbrak}[1]{\ensuremath{\left.#1\right)}}
\providecommand{\cbrak}[1]{\ensuremath{\left\{#1\right\}}}
\providecommand{\lcbrak}[1]{\ensuremath{\left\{#1\right.}}
\providecommand{\rcbrak}[1]{\ensuremath{\left.#1\right\}}}
\theoremstyle{remark}
\newtheorem{rem}{Remark}
\newcommand{\sgn}{\mathop{\mathrm{sgn}}}
%\providecommand{\hilbert}{\overset{\mathcal{H}}{ \rightleftharpoons}}
\providecommand{\system}{\overset{\mathcal{H}}{ \longleftrightarrow}}
%\newcommand{\solution}[2]{\textbf{Solution:}{#1}}
\newcommand{\solution}{\noindent \textbf{Solution: }}
\newcommand{\cosec}{\,\text{cosec}\,}
\providecommand{\dec}[2]{\ensuremath{\overset{#1}{\underset{#2}{\gtrless}}}}
\newcommand{\myvec}[1]{\ensuremath{\begin{pmatrix}#1\end{pmatrix}}}
\newcommand{\mydet}[1]{\ensuremath{\begin{vmatrix}#1\end{vmatrix}}}
\numberwithin{equation}{subsection}
\makeatletter
\@addtoreset{figure}{problem}
\makeatother
\let\StandardTheFigure\thefigure
\let\vec\mathbf
\renewcommand{\thefigure}{\theproblem}
\def\putbox#1#2#3{\makebox[0in][l]{\makebox[#1][l]{}\raisebox{\baselineskip}[0in][0in]{\raisebox{#2}[0in][0in]{#3}}}}
     \def\rightbox#1{\makebox[0in][r]{#1}}
     \def\centbox#1{\makebox[0in]{#1}}
     \def\topbox#1{\raisebox{-\baselineskip}[0in][0in]{#1}}
     \def\midbox#1{\raisebox{-0.5\baselineskip}[0in][0in]{#1}}
\vspace{3cm}
\title{Assignment 8}
\author{Subhasish Saikia\\AI20MTECH14001}
\maketitle
\begin{abstract}
This document explains the condition for diagonalization of a matrix.
\end{abstract}
\bigskip
\renewcommand{\thefigure}{\theenumi}
\renewcommand{\thetable}{\theenumi}
Download latex-tikz codes from
\begin{lstlisting}
https://github.com/subhasishsaikia22/EE5609-Matrix-theory
\end{lstlisting}
\section{\textbf{Problem}}
Which of the following matrices is not diagonalizable over $\mathbb{R}$?
\begin{enumerate}
\item $\myvec{2&0&1\\0&3&0\\0&0&2}$\\ 
\item $\myvec{1&1\\1&1}$\\ 
\item $\myvec{2&1&0\\0&3&0\\0&0&3}$\\ 
\item $\myvec{1&-1\\2&4}$
\end{enumerate}
\section{\textbf{Solution}}
\renewcommand{\thetable}{1}
\begin{longtable}{|l|l|}
\hline
\text{Definition} & \text{Let $\vec{W_i}$ be the eigenspace corresponding to eigenvalue $\lambda_i$ of a $n$ by $n$ matrix $\vec{A}$.} \\
& \text{then the characteristic polynomial of $\vec{A}$ can be expressed as:}\\ & \parbox{10cm} {\begin{align}
    \brak{x-\lambda_1}^{d_1}\brak{x-\lambda_2}^{d_2}...\brak{x-\lambda_k}^{d_k}
\end{align}}\\
& \text{where ${d_i}$ is the algebraic multiplicity of eigenvalue $\lambda_i$} and\\
& \text{dim$\myvec{W_i}$is the geometric multiplicity of the corresponding eigenvalue$\lambda_i$.}\\
& \text{Matrix $\vec {A}$ is said to be diagonalizable if and only if}\\& \parbox{10cm} {\begin{align}
    dim\myvec{W_i}= {d_i}
    \end{align}}\\
& \text{i.e geometric multiplicity=algebraic multiplicity}\\
\hline
\text{Explanation}  & \text{dim$\myvec{W_i}$, the geometric multiplicity of the corresponding eigen value $\lambda_i$ for a matrix $\vec{A}$ }\\
& \text{is the null space of the matrix \brak{A-\lambda_iI} which is the nullity of the  matrix \brak{A-\lambda_iI}}  \\
& \text{From Rank nullity theorem,}\\
& \text{Nullity of a matrix= no of variables$-$ Rank of the matrix.}\\
\hline
\textbf{Option 1} & \text{$\myvec{2&0&1\\0&3&0\\0&0&2}$}\\
\hline
\textbf{solution} & \text{The given matrix is a $3$ by $3$ triangular matrix and as such the eigen values of the matrix are:}\\  & \parbox{10cm} {\begin{align}
\lambda_1=2,\lambda_2=3 \quad and \quad \lambda_3=2
\end{align}}\\
& \text{The characteristic polynomial of the matrix:}\\& \parbox{10cm} {\begin{align}
    \brak{x-2}^{2}\brak{x-3}^{1}
\end{align}}\\
& \text{Thus the algebraic multiplicity ${d_1}$  of eigenvalue=$2$ is $2$.}\\
& \text{For $\lambda_1=2$}\\& \parbox{10cm} 
{\begin{align}
\brak{A-\lambda_1I}=\brak{A-2I}=\myvec{2-2&0&1\\0&3-2&0\\0&0&2-2}\\
\implies \myvec{0&0&1\\0&1&0\\0&0&0}
\end{align}}\\
& \text{Thus the rank of $\brak{A-2I}$ is $2$ and nullity of $\brak{A-2I}$, dim$\myvec{W_1}$ is $3-2$=$1$}\\
& \text{Since dim$\myvec{W_1}$\neq${d_1}$,}\\
& \parbox{10cm}{\begin{center}
\textbf{The given matrix is not diagonalizable}
\end{center}}\\
\hline 
\textbf{Option 2} & \text{$\myvec{1&1\\1&1}$}\\ 
\hline
\textbf{solution} & \text{The given matrix is a $2$ by $2$ matrix and as such the eigen values of the matrix are:}\\  & \parbox{10cm} {\begin{align}
\mydet{\vec{A}-\lambda I}=0\\
\implies \lambda^2-2\lambda=0\\
\implies \lambda_1=0 \quad and \quad \lambda_2=2
\end{align}}\\
& \text{The characteristic polynomial of the matrix:}\\& \parbox{10cm} {\begin{align}
    x\brak{x-2}
\end{align}}\\
& \text{Thus the algebraic multiplicity ${d_2}$ of the matrix is $1$}\\
& \text{For $\lambda_2=2$}\\& \parbox{10cm} 
{\begin{align}
\brak{A-\lambda_2I}=\brak{A-2I}=\myvec{1-2&1\\1&1-2}\\
\implies \myvec{-1&1\\1&-1}\xleftrightarrow[R_2=R_2+R_1]\ \myvec{-1&1\\0&0}
\end{align}}\\
& \text{Thus the rank of $\brak{A-2I}$ is $1$ and nullity of $\brak{A-2I}$, dim$\myvec{W_2}$ is $2-1$=$1$}\\
& \text{Since dim$\myvec{W_2}={d_2}$,}\\
& \parbox{10cm}{\begin{center}
\textbf{The given matrix is diagonalizable}
\end{center}}\\
\hline 
\textbf{Option 3} & \text{$\myvec{2&1&0\\0&3&0\\0&0&3}$}\\
\hline
\textbf{solution} & \text{The given matrix is a $3$ by $3$ triangular matrix and as such the eigen values of the matrix are:}\\  & \parbox{10cm} {\begin{align}
\lambda_1=2,\lambda_2=3  \quad and \quad \lambda_3=3
\end{align}}\\
& \text{The characteristic polynomial of the matrix:}\\& \parbox{10cm} {\begin{align}
    \brak{x-2}\brak{x-3}^{2}
\end{align}}\\
& \text{Thus the algebraic multiplicity ${d_2}$ of the matrix is $2$}\\
& \text{For $\lambda_2=3$}\\& \parbox{10cm} 
{\begin{align}
\brak{A-\lambda_2I}=\brak{A-3I}=\myvec{2-3&1&0\\0&3-3&0\\0&0&3-3}\\
\implies \myvec{-1&1&0\\0&0&0\\0&0&0}
\end{align}}\\
& \text{Thus the rank of $\brak{A-3I}$ is $1$ and nullity of $\brak{A-3I}$, dim$\myvec{W_2}$ is $3-1$=$2$}\\
& \text{Since dim$\myvec{W_2}={d_2}$,}\\
& \parbox{10cm}{\begin{center}
\textbf{The given matrix is diagonalizable}
\end{center}}\\
\hline 
\textbf{Option 4} & \text{$\myvec{1&-1\\2&4}$}\\ 
\hline
\textbf{solution} & \text{The given matrix is a $2$ by $2$ matrix and as such the eigen values of the matrix are:}\\  & \parbox{10cm} {\begin{align}
\mydet{\vec{A}-\lambda I}=0\\
\implies \lambda^2-5\lambda+6=0\\
\implies \lambda\brak{\lambda-3}-2\brak{\lambda-3}=0\\
\implies \lambda_1=2 \quad and \quad \lambda_2=3
\end{align}}\\
& \text{The characteristic polynomial of the matrix:}\\& \parbox{10cm} {\begin{align}
    \brak{x-2}\brak{x-3}
\end{align}}\\
& \text{Thus the algebraic multiplicity ${d_2}$ of the matrix is $1$}\\
& \text{For $\lambda_2=3$}\\& \parbox{10cm} 
{\begin{align}
\brak{A-\lambda_2I}=\brak{A-3I}=\myvec{1-3&-1\\2&4-3}\\
\implies \myvec{-2&-1\\2&1}\xleftrightarrow[R_2=R_2+R_1]\ \myvec{-2&-1\\0&0}
\end{align}}\\
& \text{Thus the rank of $\brak{A-3I}$ is $1$ and nullity of $\brak{A-3I}$, dim$\myvec{W_2}$ is $2-1$=$1$}\\
& \text{Since dim$\myvec{W_2}={d_2}$,}\\
& \parbox{10cm}{\begin{center}
\textbf{The given matrix is diagonalizable}
\end{center}}\\
\hline
\caption{Explanation}
\label{table:1}
\end{longtable}
\end{document}
