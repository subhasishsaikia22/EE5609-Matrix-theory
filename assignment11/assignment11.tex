\documentclass[journal,12pt]{IEEEtran}
\usepackage{longtable}
\usepackage{setspace}
\usepackage{gensymb}
\singlespacing
\usepackage[cmex10]{amsmath}
\newcommand\myemptypage{
\null
\thispagestyle{empty}
\addtocounter{page}{-1}
\newpage
}
\usepackage{amsthm}
\usepackage{mdframed}
\usepackage{mathrsfs}
\usepackage{txfonts}
\usepackage{stfloats}
\usepackage{bm}
\usepackage{cite}
\usepackage{cases}
\usepackage{subfig}

\usepackage{longtable}
\usepackage{multirow}

\usepackage{enumitem}
\usepackage{mathtools}
\usepackage{steinmetz}
\usepackage{tikz}
\usepackage{circuitikz}
\usepackage{verbatim}
\usepackage{tfrupee}
\usepackage[breaklinks=true]{hyperref}
\usepackage{graphicx}
\usepackage{tkz-euclide}

\usetikzlibrary{calc,math}
\usepackage{listings}
    \usepackage{color}                                            %%
    \usepackage{array}                                            %%
    \usepackage{longtable}                                        %%
    \usepackage{calc}                                             %%
    \usepackage{multirow}                                         %%
    \usepackage{hhline}                                           %%
    \usepackage{ifthen}                                           %%
    \usepackage{lscape}    
\usepackage{multicol}
\usepackage{chngcntr}

\DeclareMathOperator*{\Res}{Res}

\renewcommand\thesection{\arabic{section}}
\renewcommand\thesubsection{\thesection.\arabic{subsection}}
\renewcommand\thesubsubsection{\thesubsection.\arabic{subsubsection}}

\renewcommand\thesectiondis{\arabic{section}}
\renewcommand\thesubsectiondis{\thesectiondis.\arabic{subsection}}
\renewcommand\thesubsubsectiondis{\thesubsectiondis.\arabic{subsubsection}}


\hyphenation{op-tical net-works semi-conduc-tor}
\def\inputGnumericTable{}                                 %%

\lstset{
%language=C,
frame=single,
breaklines=true,
columns=fullflexible
}
\begin{document}
\onecolumn

\newtheorem{theorem}{Theorem}[section]
\newtheorem{problem}{Problem}
\newtheorem{proposition}{Proposition}[section]
\newtheorem{lemma}{Lemma}[section]
\newtheorem{corollary}[theorem]{Corollary}
\newtheorem{example}{Example}[section]
\newtheorem{definition}[problem]{Definition}

\newcommand{\BEQA}{\begin{eqnarray}}
\newcommand{\EEQA}{\end{eqnarray}}
\newcommand{\define}{\stackrel{\triangle}{=}}
\bibliographystyle{IEEEtran}
\raggedbottom
\setlength{\parindent}{0pt}
\providecommand{\mbf}{\mathbf}
\providecommand{\pr}[1]{\ensuremath{\Pr\left(#1\right)}}
\providecommand{\qfunc}[1]{\ensuremath{Q\left(#1\right)}}
\providecommand{\sbrak}[1]{\ensuremath{{}\left[#1\right]}}
\providecommand{\lsbrak}[1]{\ensuremath{{}\left[#1\right.}}
\providecommand{\rsbrak}[1]{\ensuremath{{}\left.#1\right]}}
\providecommand{\brak}[1]{\ensuremath{\left(#1\right)}}
\providecommand{\lbrak}[1]{\ensuremath{\left(#1\right.}}
\providecommand{\rbrak}[1]{\ensuremath{\left.#1\right)}}
\providecommand{\cbrak}[1]{\ensuremath{\left\{#1\right\}}}
\providecommand{\lcbrak}[1]{\ensuremath{\left\{#1\right.}}
\providecommand{\rcbrak}[1]{\ensuremath{\left.#1\right\}}}
\theoremstyle{remark}
\newtheorem{rem}{Remark}
\newcommand{\sgn}{\mathop{\mathrm{sgn}}}
%\providecommand{\hilbert}{\overset{\mathcal{H}}{ \rightleftharpoons}}
\providecommand{\system}{\overset{\mathcal{H}}{ \longleftrightarrow}}
%\newcommand{\solution}[2]{\textbf{Solution:}{#1}}
\newcommand{\solution}{\noindent \textbf{Solution: }}
\newcommand{\cosec}{\,\text{cosec}\,}
\providecommand{\dec}[2]{\ensuremath{\overset{#1}{\underset{#2}{\gtrless}}}}
\newcommand{\myvec}[1]{\ensuremath{\begin{pmatrix}#1\end{pmatrix}}}
\newcommand{\mydet}[1]{\ensuremath{\begin{vmatrix}#1\end{vmatrix}}}
\numberwithin{equation}{subsection}
\makeatletter
\@addtoreset{figure}{problem}
\makeatother
\let\StandardTheFigure\thefigure
\let\vec\mathbf
\renewcommand{\thefigure}{\theproblem}
\def\putbox#1#2#3{\makebox[0in][l]{\makebox[#1][l]{}\raisebox{\baselineskip}[0in][0in]{\raisebox{#2}[0in][0in]{#3}}}}
     \def\rightbox#1{\makebox[0in][r]{#1}}
     \def\centbox#1{\makebox[0in]{#1}}
     \def\topbox#1{\raisebox{-\baselineskip}[0in][0in]{#1}}
     \def\midbox#1{\raisebox{-0.5\baselineskip}[0in][0in]{#1}}
\vspace{3cm}
\title{Assignment 11}
\author{Subhasish Saikia\\AI20MTECH14001}
\maketitle
\begin{abstract}
This document explains the transformation of vector space $\mathbb{P}_{n}$
\end{abstract}
\bigskip
\renewcommand{\thefigure}{\theenumi}
\renewcommand{\thetable}{\theenumi}
Download latex-tikz codes from
\begin{lstlisting}
https://github.com/subhasishsaikia22/EE5609-Matrix-theory
\end{lstlisting}
\section{\textbf{Problem}}
Consider the vector space $\mathbb{P}_{n}$ of real polynomials in ${x}$ of degree less than or equal to ${n}$. Define $\vec{T} : \mathbb{P}_{2}\rightarrow \mathbb{P}_{3}$ by $\vec{T}(f(x))=\int\limits_0^xf(t)\,dt+f'(x)$.Then the matrix  representation of \vec{T} with respect to the bases $\cbrak{1,x,x^2}$ and $\cbrak{1,x,x^2,x^3}$ is \\
\begin{enumerate}
\item  $\myvec{0&1&0&0\\1&0&\frac{1}{2}&0\\0&2&0&\frac{1}{3}}$\\\\
\item $\myvec{0&1&0\\1&0&2\\0&\frac{1}{2}&0\\0&0&\frac{1}{3}}$ \\\\
\item  $\myvec{0&1&0&0\\1&0&2&0\\0&\frac{1}{2}&0&\frac{1}{3}}$ \\\\
\item $\myvec {0&1&0\\1&0&\frac{1}{2}\\0&2&0\\0&0&\frac{1}{3}}$ 
\end{enumerate}

\section{\textbf{Solution}}
\renewcommand{\thetable}{2}
\begin{longtable}{|l|l|}
\hline
\textbf{Given} & \parbox{13cm} {\begin{align}
 \vec{T}(f(x))=\int\limits_0^xf(t)\,dt+f'(x)
\end{align}}\\
&\text{The basis vector of domain space $\cbrak{1,x,x^2} $.}\\ &\text{The basis vector of co-domain space $\cbrak{1,x,x^2,x^3} $. }\\ 
&\text{For the Transformation, we first find the images of the basis vector of domain space}\\ &\text {with respect to \vec{T}, which is then expressed as the linear combination of basis vector}\\&\text{of codomain space basis.}
\hline
\textbf{solution}  &\text{Base vector 1:}\\ & \parbox{13cm} {\begin{align}
f(x)=1 \implies f'(x)=0\\
\vec{T}(1)=\int\limits_0^x1\,dt+0=\sbrak{t}\limits_0^x=x\\
=0+1.x+0.x^2+0.x^3\\
\vec{T}(1)=\myvec{0\\1\\0\\0}
\end{align}}\\
 &\text{Base vector 2:}\\ & \parbox{13cm} {\begin{align}
f(x)=x \implies f'(x)=1\\
\vec{T}(x)=\int\limits_0^xt\,dt+1=\sbrak{\frac{t^2}{2}}\limits_0^x+1=\frac{x^2}{2}+1\\
=1+0.x+\frac{1}{2}.x^2+0.x^3\\
\vec{T}(x)=\myvec{1\\0\\\frac{1}{2}\\0}
\end{align}}\\
&\text{Base vector 3:}\\ & \parbox{13cm} {\begin{align}
f(x)=x^2 \implies f'(x)=2x\\
\vec{T}(x^2)=\int\limits_0^xt^2\,dt+2t=\sbrak{\frac{t^3}{3}}\limits_0^x+\sbrak{2t}\limits_0^x=\frac{x^3}{3}+2x\\
=0+2.x+0.x^2+\frac{1}{3}.x^3\\
\vec{T}(x^2)=\myvec{0\\2\\0\\\frac{1}{3}}
\end{align}}\\
&\text{Therefore}\\ & \parbox{13cm} {\begin{align}
\vec{T}=\myvec{\vec{T}(1)& \vec{T}(x)& \vec{T}(x^2)}\\
=\myvec{0&1&0\\1&0&2\\0&\frac{1}{2}&0\\0&0&\frac{1}{3}}
\end{align}}\\
\hline
\caption{Solution}
\label{deftab}
\end{longtable}\\
\section{\textbf{option}}
\renewcommand{\thetable}{3}
\begin{longtable}{|l|l|l|}
\hline
\textbf{Option}&\textbf{Solution}&\textbf{True/}\\&&\textbf{False}\\
\hline
1 & $\myvec{0&1&0&0\\1&0&\frac{1}{2}&0\\0&2&0&\frac{1}{3}}$&False
\hline
2 & $\myvec{0&1&0\\1&0&2\\0&\frac{1}{2}&0\\0&0&\frac{1}{3}}$&True
\hline
3 &$\myvec{0&1&0&0\\1&0&2&0\\0&\frac{1}{2}&0&\frac{1}{3}}$&False
\hline
4 &$\myvec {0&1&0\\1&0&\frac{1}{2}\\0&2&0\\0&0&\frac{1}{3}}$&False
\hline
\caption{correct option}
\label{deftab}
\end{longtable}
\end{document}
